\documentclass[ebook,9pt,oneside,openany]{memoir}
\usepackage[utf8x]{inputenc}
\usepackage[english]{babel}
\usepackage{url}
\usepackage[margin=0.4in]{geometry}

% for placeholder text
\usepackage{lipsum}

\title{La Suisse comme paradis fiscal au XXe siècle}
\author{Yann Bolliger, Pietro Carta, Romain Mendez}

\begin{document}
\maketitle

La place financière suisse a vu une énorme croissance presque non perturbée tout au long du XXème siècle. Cela a été possible grâce à la neutralité et la stabilité de la Suisse notamment en période de guerre mais surtout aussi grâce au «secret bancaire». Ce dernier était déjà une pratique des banques suisses en XIXème siècle quand les grandes banques commencent à dominer la place financière suisse. Ces banques-là profitent considérablement des afflux de capitaux étrangers. Pendant la Grande Guerre les banques utilisent le secret bancaire pour attirer les capitaux étrangers fuyant de lourdes fiscalités implémentées par les pays en guerre.

Cela leur permet de devenir une force majeur à l’échelle de la finance mondiale ainsi qu’une qu’une influence principale dans la politique nationale.
En effet, l’influence des banques dans la politique fédérale étaient tellement grande que le secret bancaire fut renforcé par la loi sur les banques en 1934, sans susciter de grands débats au parlement. 

Avec la seconde guerre mondiale, de nouveau, la place financière Suisse profite de la fuite de capitaux étrangers provenant de pays en guerre. Sous couvert de la neutralité, les banques suisses arrivent à maintenir des liens très proches avec tous les belligérants, mais surtout avec les forces de l’Axe. Ce qui mène la suisse dans une grande isolation diplomatique à la fin de la guerre. Par exemple, les États-unis gelèrent les avoirs des banques suisses déposés en Amérique déjà en 1941. Néanmoins la diplomatie Suisse obtient le maintien du secret bancaire contre les revendications des vainqueurs. Cela marque le début d’une période de croissance sans précédent pour la place financière pendant les «trente glorieuses».

Mais le secret bancaire a toujours été fortement critiqué – les plus importants critiques étant les États-unis et la France. Dans la deuxième partie du XXe siècle, la diplomatie américaine obtient de la Suisse quelques concessions qui n’ont toutefois aucun impact. Les critiques se renforcèrent tout au long des années 1970 avec un grand nombre de scandals dans lesquels étaient impliquées les grandes banques suisses. Cela a entraîné la création de la déclaration de Berne. Cette organisation a lancé une initiative populaire contre le secret bancaire qui a été refusée très clairement en 1984.

Les grandes banques et les élites politiques ont ainsi réussi à maintenir ce statut privilégié de la place financière pendant plus que 50 ans. Ils l’ont défendu contre la pression de l’intérieur et de l’extérieur et ce n’est seulement après la crise financière en 2007 que le secret sera levé.

Dans le cadre de notre recherche nous essayerons de retrouver ces événements dans la presse romande. Celle-ci étant plutôt proche des cercles financiers – surtout le «journal de Genève» –, nous évaluerons aussi leurs positions sur le secret bancaire et si cette proximité peut-être confirmée par les articles du corpus. Afin de nous demander, comment évolue la couverture médiatique du secret bancaire au XXe siècle ?
\section*{Information Bibliographiques et de Corpus}
Nous admettons dans notre analyse les articles extrait de la “Gazette de Lausanne” et du “Journal de Genève”, issus dans les deux journaux pendants la période 1900-1995. Pour restreindre l’analyse aux article pertinents, le corpus d’articles des deux journaux sera filtré de façon tel que les articles contiennent des mots clés, repérés à travers l’analyse d’autres sources primaires et secondaires.

Les sources primaire que nous analysons, outres que les archives du Temps, sont de nature politique, juridiques, ou diplomatique.

L’organisation tiers-mondiste “Déclaration de Berne” en collaboration avec le Parti Socialiste publie en 1978 le pamphlet “Les Secrets du secrets bancaire suisse” CIT où les conséquences internationales et intérieures du secret bancaire sont dénoncé. Cet ouvrage nous expose au discour qui entourait le sujet pendants les années 70s et 80s. Un nombre de scandale pertinents au secret bancaire en sont les protagonistes.

Les sources juridiques nous témoignent un conflit entre la Suisse et pays étranger en fait de secret bancaire. La sentence “CIT tribunal federal” en faveur du maintien du secret bancaire en 1969 nous montre que la loi est appliqué avec véhémence.

Les États Unis rejoindrons enfin une entente avec la Suisse sur le secret bancaire, comme témoigné par (CIT memorandum on insider trading) et (CIT 1974 entente).

L’histoire financière suisse a été étudié extensivement par Sébastien Guex et Malik Mazbouri.
Les aspects juridiques du secret bancaire ont été étudié en 1969 par Mueller. La spécificité du cas suisse à niveau international est analysé par Meier et al. (TOUT CITER)

Mots clés: secret bancaire, place financière suisse, banques suisses, forfait fiscal, liechtenstein, impôt anticipé, pots-de-vin, manipulation boursières, paradis fiscal, compensateurs, corruption, affaire chiasso, argent sale, blanchiment.

\section*{Outil Méthodologiques}
Afin de pouvoir traiter en un temps raisonnable notre corpus de texte, plusieurs pistes d’analyse s’offrent à nous. Dans un premier temps un filtrage des articles s’impose, pour ne travailler que sur des articles contenant des termes importants du sujet (voir la liste de terme clés dans la précédente partie).

    Ensuite, nous pensons procéder à des visualisations du corpus filtré avec des logiciels tel que Iramuteq afin de voir les relations entre les mots (quels mots se suivent souvent, et quelle est leur connotation). Nous voulons aussi nous intéresser aux outils d’analyse de texte plus récents qui peuvent permettre par exemple de dégager une impression du texte (s’il est plutôt positif ou non), afin de tenter de pouvoir visualiser l’opinion de la rédaction sur les articles qui abordent le sujet au fil du temps.
    
    Cependant cette analyse de données devra être contrôlé en permanence car ce genre d’outils ne sont pas totalement précis 100\% du temps (ni l’être humain d’ailleurs).
    
    Nous pensons donc utiliser les sources secondaires pour forger une attente du corpus, ceux à quoi nous nous attendons après ce type d’analyse. C’est à dire par exemple utiliser les informations recueillies sur les rédactions des journaux dans la conférence du 31 Octobre “Un parcours singulier dans l'histoire de la presse romande: "Gazette de Lausanne" (1798-1991) et "Journal de Genève (1826-1998)” par Prof. Alain Clavien afin de vérifier nos résultats avec les différentes tendances des rédactions de ces deux journaux. Cela devrait pouvoir nous servir de garde-fou sur nos résultats. Une fois ce travail accompli, nous nous servirons des résultats pour analyser la position idéologique des journaux au fil du temps dans la période étudiée.
\end{document}